\chapter*{Introduction}
\addcontentsline{toc}{chapter}{Introduction}
J'ai effectué un stage d'une durée de quatre mois au sein du département Web Sémantique de Logilab : une société spécialisée dans le développement de solutions informatique pour la gestion de connaissances et d'applications scientifiques. Logilab développe et utilise CubicWeb, un framework de web sémantique distribué sous la licence libre \glslink{lgpl}{LGPL} écrit en langage Python. 

Ces dernières années les entreprises proposent de plus en plus leur services via des applications web (\textit{\glslink{saas}{Software As A Service}}) et utilisent l'informatique dans les nuages (\textit{cloud computing}) pour héberger ces applications. Les services en ligne doivent offrir des temps de réponse courts afin de satisfaire les utilisateurs. Le développement de ces services doit respecter les principes du web et principalement l'architecture \glslink{REST}{REST}. REST est un style d'architecture pour les systèmes hypermédia et est très adapté au World Wide Web. L'architecture REST définit des contraintes qui permettent, si elle sont respectées, d'avoir des applications web performantes, scalable et fiable.

Le sujet de mon stage porte sur la mise en place d'une architecture REST dans le framework CubicWeb. Dans un premier temps il s'agit de dresser l'état des lieux du respect des contraintes de cette l'architecture au sein du CubicWeb. Dans un deuxième temps refondre le code nécessaire pour respecter ces contraintes mais sans introduire de gros changement dans l'utilisation du CubicWeb. Enfin vérifier le fonctionnement des applications existantes et assurer qu'aucune régression n'a été introduite par le nouveau code.

J'ai choisi ce stage car le sujet s'inscrit parfaitement dans la logique de mon projet professionnel. En effet, mon objectif est de travailler dans le développement d'application d'Internet rsiche. Ces d'application doivent avoir des temps de réponse courts. L'architecture REST est une solution pour cette contrainte. Le fait de travailler sur des projets libres m'a aussi motivé pour effectuer ce stage.

Ce rapport de stage détaille le travail effectué pour atteindre les objectifs fixé au début de stage. Je commencerai par présenter le contexte de stage : l'entreprise d'accueil, CubicWeb et l'architecture REST. Ensuite je détaillerai la méthode de travail utilisée durant le période de stage. Je présenterai ensuite le travail effectué pour atteindre les objectifs de stage. Enfin Je conclurai ce rapport par dresser le bilan des résultats obtenus.