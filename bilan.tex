\chapter{Bilan}
\section{Bilan des résultats}
Le travaille effectué pendant ce stage a permis de dresser l'état des lieux du respect des contraintes de l'architecture \glslink{REST}{REST} dans CubicWeb. Un document \glslink{cwep}{CWEP} a été rédigé qui détaillent les points de CubicWeb ne respectant pas REST. Le document classe les applications CubicWeb selon le modèle développé par Leonard Richardson\cite{rmm}. Ce modèle permet de classer une application selon quatre niveaux (de 0 à 3) et définit comment passer d'un niveau à un autre. Le niveau 3 est une condition nécessaire pour réaliser une application respectant complètement l'architecture REST telle qu'elle est définie par Roy Fielding. Je n'ai pas eu beaucoup de retours sur le contenu de ce document, certains points reste des questions ouvertes. Les documents CWEP sont utilisés par les développeurs de CubicWeb pour implanter de nouvelles évolutions de framework. Ces points vont être alors abordés dans l'avenir et des solutions peuvent être implantés pour les corriger. 

J'ai pu proposer des solutions pour certains points. J'ai implanté une solution qui permettait de supprimer le mécanisme de \textit{breadcrumbs} utilisé dans CubicWeb. Cette solution permet de réduire la quantité de donnés stockée sur le serveur et le nombre de traitements effectués par celui-ci. J'ai aussi implanté dans Pyramid-CubicWeb des vues permettant de faire les opérations \glslink{crud}{CRUD} en respectant REST. La solution fonctionne dans CubicWeb lorsque celui-ci est utilisé avec Pyramid-CubicWeb. 

Enfin j'ai implanté la négociation de contenu qui va permettre au client d'une application CubicWeb de choisir le format de données à utiliser via l'entête HTTP \texttt{Accept}. Cette solution respecte bien les spécification de protocole HTTP et ainsi les principes de REST. 

Durant le stage, j'ai participé aux VSprint et j'ai contribué à améliorer les outils interne de Logilab. J'ai développé une solution permettant d'avoir la différence entre la dernière version d'une révision et une version obsolète.     
\subsection{Atteinte des objectifs}
Dans l'ensemble les objectifs qui ont été fixés au début de stage ont été atteints. Le premier objectif de stage qui est de dresser l'état des lieux du respect des principes REST dans CubicWeb a été atteint. Un document détaillant l'état des lieux a été réalisé et peut être utilisé pour faire évoluer CubicWeb. 

Les deuxième objectif qui est de refondre dans CubicWeb le code nécessaire pour respecter les principes de REST n'est pas complètement atteint. La solution intégrée ne permet pas de transférer complètement l'état des requêtes vers le client. CubicWeb stocke toujours des données relatives à chaque requête. La solution qui permet de transférer complètement l'état vers le client introduit un grand changement du l'utilisation de l'interface web de Cubicweb. 

Le troisième objectif a été atteint puisque les applications existantes fonctionne encore avec le code écrit.

\subsection{Perspective d'évolution}
La solution proposée pour réaliser des opérations \glslink{crud}{CRUD} sur les entités peut être améliorer et respecter les contraintes de niveau 3 du modèle de Leonard Richardson. L'évolution possible est de rajouter des contrôles Hypermédia au réponses retournées. Le principe des contrôles Hypermédia est d'inclure des URLs pour indiquer les opérations qui peuvent être faites sur la ressource retournée. 

\section{Bilan personnel}
Ce stage est très enrichissant sur plan personnel. Il m'a permis de mettre en application les connaissances que j'ai acquises durant mon cursus universitaire. Il m'a aussi permis d'acquérir de nouvelles connaissances techniques. Ce stage était l'occasion pour moi de contribuer à un projet sous licence libre. Grâce à cette expérience j'ai acquis les bonnes pratiques pour contribuer de manière efficace aux projets libres.  
\subsection{Connaissances acquises}
Ce stage est aussi enrichissant en terme de connaissances acquises. J'ai découvert deux framework web : CubicWeb et Pyramid. CubicWeb étant un framework de web sémantique m'a permis d'avoir une ouverture vers ce domaine. 

Ce stage était l'occasion pour moi d'améliorer mes connaissances des langages Python et JavaScript. J'ai pu découvrir des fonctionnalités avancés de ces deux langages. J'ai appris à utiliser la librairie JavaScript JQuery qui est très utilisée dans le développement web. 

Durant le stage j'ai appris à utiliser le gestionnaire de version Mercurial. J'ai été amené à utiliser les fonctionnalité de réécriture de l'historique de celui-ci. En effet, il était indispensable de modifier les révisions envoyées sur l'entrepôt dans le but d'avoir un historique propre. Cette année, j'ai suivi les cours sur l'utilisation de Git mais les cours ne parlait pas de cette fonctionnalité que je trouve très indispensable. 

J'ai appris à utiliser plusieurs outils comme l'éditeur VIM et j'ai découvert Salt, un outil de gestion de configuration. 
\subsection{Projet professionnel}
J'ai décidé de faire ce stage car il s'inscrit parfaitement dans la logique de mon projet professionnel qui est de travailler dans le domaine de web pour la réalisation d'applications d'Internet riches. Ce stage m'as permet d'acquérir de bonnes connaissances dans ce domaine.  
