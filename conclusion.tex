\chapter*{Conclusion}
\addcontentsline{toc}{chapter}{Conclusion}
Ce stage que j'ai effectué à Logilab a permis de montrer les différentes implantations de CubicWeb ne respectant pas les principes de l'architecture REST. Un document a été réalisé détaillant ces points et liste plusieurs évolutions possibles de CubicWeb. J'ai eu l'occasion de corriger certains de ces point mais il reste beaucoup de travail à faire pour mettre en place une architecture REST dans CubicWeb.     

Ce stage m'a permis de découvrir comment sont appliquées les méthodes agiles pour les projets informatiques
dans l'industrie. J'ai vu aussi comment il est possible d'adapter plusieurs méthodes agiles pour les besoins de l'entreprise.

Ce stage est très enrichissant sur le plan techniques. Cela s'explique par le fait que Logilab est très experte dans plusieurs domaine et met en place un système efficace permettant le partage des connaissances entre l'ensemble de ses employés. J'ai eu l'occasion d'approfondir mes connaissances sur l'architecture REST. Durant ce stage j'ai étudié REST en partant de zéro et en utilisant comme base la thèse de Roy Fielding le créateur de REST. Ceci m'a permis de bien comprendre les contraintes définies dans REST. 

J'ai aussi amélioré mes connaissances des langages Python et JavaScript et durant ce stage j'ai pu découvrir les fonctionnalités avancés de ces deux langages. 

Pour conclure, ce stage était une expérience très enrichissante qui va me servir dans la suite de mes études mais aussi dans le cadre professionnel.
