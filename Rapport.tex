\documentclass[a4paper,12pt,dvipsnames]{report}
\usepackage[utf8]{inputenc}
\usepackage[T1]{fontenc}
\usepackage[nottoc,numbib]{tocbibind}
\usepackage[french, english]{babel}
\usepackage{csquotes}
\usepackage{lmodern}
\usepackage{lipsum}
\usepackage{graphicx}
\usepackage{microtype}
\usepackage[frenchb]{translator}
\usepackage[a4paper]{geometry}
\usepackage{tikz} 
\usepackage{standalone}
\usepackage{parskip}
\usepackage{lscape}
\usepackage{pgfgantt}

\usepackage{color, colortbl}
\usepackage{minted}

\usepackage[backend=biber,sorting=none,alldates=short]{biblatex}
\addbibresource{rapport.bib}
\usepackage{hyperref}
\hypersetup{
    colorlinks=true,
  	citecolor=Violet,
  	linkcolor=blue,
  	urlcolor=black,
    pdfborder={0 0 0},
}

%%%%%%%%%%%%%%%%%%%%%%%%%%%%%%%%%%%%%%%%%%%%%%%%%%%%%%%%%%%%%%%%
\usepackage[style=super,
			toc, nopostdot,
			xindy={language=french, codepage=utf8},
			acronym,
			]{glossaries}
\makeglossaries


%%%%%%%%%%%%%%%%%%%%%%%%%%%%%%%%%%%%%%%%%%%%%%%%%%%%%%%%%%%%%%%%
\newacronym{REST}{REST}{REpresentational State Transfer}
\newacronym{yams}{YAMS}{Yet Another Magic Schema}
\newacronym{rql}{RQL}{Relation Query Language}
\newacronym{xml}{XML}{JavaScript Object Notation}
\newacronym{rdf}{RDF}{Resource Description Framework}
\newacronym{json}{JSON}{JavaScript Object Notation}
\newacronym{saas}{SAAS}{Software As A Service}
\newacronym{sql}{SQL}{Structured Query Language}
\newacronym{ldap}{LDAP}{Lightweight Directory Access Protocol}
\newacronym{lgpl}{LGPL}{GNU Lesser General Public License}
\newacronym{html}{HTML}{Hypertext Markup Language}
\newacronym{http}{HTTP}{Hypertext Transfer Protocol}
\newacronym{xp}{XP}{eXtreme Programming}
\newacronym{tdd}{TDD}{Test Driven Developpement}
\newacronym{pep}{PEP}{Python Enhancement Proposals}
\newacronym{cwep}{CWEP}{CubicWeb Evolution Proposal}
\newacronym{ajax}{AJAX}{Asynchronous JavaScript and XML}
\newacronym{crud}{CRUD}{Create, Read, Update, Delete}
\newacronym{hateoasac}{HATEOAS}{Hypermedia as the Engine of Application State}
\newacronym{jsonld}{JSON-LD}{JavaScript Object Notation for Linked Data}

%%%%%%%%%%%%%%%%%%%%%%%%%%%%%%%%%%%%%%%%%%%%%%%%%%%%%%%%%%%%%%%%%








%%%%%%%%%%%%%%%%%%%%%%%%%%%%%%%%%%%%%%%%%%%%%%%%%%%%%%%%%%%%%%%%%

\newlength\marginwidth
\setlength\marginwidth{\dimexpr\marginparwidth+\marginparsep}
\newlength\totalwidth 
\setlength\totalwidth{\dimexpr\textwidth+\marginwidth}

\renewcommand{\thesection}{\arabic{section}}

\newcommand{\ignore}[1]{}

\newenvironment{abstractpage}
  {\cleardoublepage\vspace*{\fill}\thispagestyle{empty}}
  {\vfill\cleardoublepage}
\renewenvironment{abstract}[1]
  {\bigskip\selectlanguage{#1}%
   \begin{center}\bfseries\abstractname\end{center}}
  {\par\bigskip}

\definecolor{Gray}{gray}{0.9}


\begin{document}
\thispagestyle{empty}

\begin{center}
\begin{minipage}[t]{0.5\textwidth}
  \begin{flushleft}
    \includegraphics [scale=.4]{images/upslogo.jpg} \\[0.5cm]
  \end{flushleft}
\end{minipage}
\begin{minipage}[t]{0.48\textwidth}
  \begin{flushright}
    \includegraphics {images/logilablogo.png} \\[0.5cm]		
  \end{flushright}
\end{minipage} \\[1.5cm]

\textbf{\underline{\large Master 1 Développement Logiciel}}

\vspace{5em}

\textsc{\Large Rapport de stage}\\[0.5cm]
\vspace{2em}
Thème:
\rule{\textwidth}{1.6pt}\vspace*{-\baselineskip}\vspace*{2pt}
\rule{\textwidth}{0.4pt}\\
\LARGE
\textbf{Mise en place d'une architecture REST dans la plateforme CubicWeb}\\
\rule{\textwidth}{0.4pt}\vspace*{-\baselineskip}\vspace{3.2pt}
\rule{\textwidth}{1.6pt}\\[\baselineskip]

\vspace{2em}

\textbf{Rabah Meradi}

\vspace{1em}
\end{center}

\normalsize
\begin{itemize}
\setlength\itemsep{0.2em}
\item [] \textbf{Maître de stage :} Sylvain Thénault
\item [] \textbf{Tuteur enseignant :} Erik Martin-Dorel
\end{itemize}


\vspace*{\fill}
\begin{center}
	{\normalsize Promotion : 2014/2015}
\end{center}

\selectlanguage{french} 
\begin{abstractpage} 

\begin{abstract}{french} 
Dans le cadre de mon Master 1 Développement Logiciel, j'ai eu l’opportunité
d'effectuer un stage d'une durée de 4 mois au sein de la société Logilab qui
est spécialisée dans le développement web sémantique et scientifique.
    
Logilab participe au développement du Logiciel Libre et met à disposition
certains de ses développements sous licence libre. Elle développe CubicWeb,
un framework pour le web sémantique écrit en langage Python qui facilite le
développement d'applications web à partir du schéma de données qu'elles manipulent 
et grâce à la réutilisation des composants appelés \textit{cubes}.

Ces dernières années il y eu a une multiplication des services en ligne
(\textit{\glslink{saas}{Software As A Service}}) ainsi que le recours à
l'informatique dans les nuage (\textit{cloud computing}). Il est devenu alors
nécessaire de respecter les principes du web dans le développement
d'applications web mais principalement l'architecture \glslink{REST}{REST}
(\textit{Representational State Transfer}). REST un style architecture pour les
systèmes hypermédia distribués très adapté au World Wide Web.

Mon stage avait comme objectif de dresser l'état des lieux du respect des
principes REST dans CubicWeb, modifier le code de celui-ci pour respecter ces
principes et vérifier le fonctionnement des applications existantes avec les
nouvelles modifications.  \end{abstract} {\textbf{\textit{Mots clés---}}
CubicWeb, REST, Web sémantique, cubes} 

\newpage 


\begin{abstract}{english} 
As part of my Master 1 Software Development, I did my four months internship
at Logilab : a company specialized in semantic web and scientific development. 
Logilab participates to the Free Software and releases some of its developments 
under a free license. It develops CubicWeb, a semantic web framework written in 
Python that ease the development of a web applications from its data schema and 
by reusing components called \textit{cubes}.

These last years there has been a growing number of online services
(\glslink{saas}{Software As A Service}) and the use of cloud computing. It then
became necessary that web applications respect the principles of the web
development, but mainly the REST architecture (Representational State Transfer).
REST is an architecture style for distributed hypermedia systems and well suited
for the World Wide Web.

The purpose of my internship is to make an inventory of CubicWeb's compliance
with \glslink{REST}{REST} principles, change the code  to make it respect these
principles and make sure that existing applications still work with the new
changes.  \end{abstract} {\textbf{\textit{Keywords---}} CubicWeb, REST, Semantic
web, cubes} \end{abstractpage}

\selectlanguage{french}

\chapter*{Remerciements}
\addcontentsline{toc}{chapter}{Remerciements}
Je tiens d’abord à remercier toutes les personnes qui m’ont apporté leur aide d’une manière ou d’une autre durant les quatre mois de mon stage.

Je voudrais remercier mon maître de stage, Monsieur Sylvain Thénault, de m’avoir accepté en tant que stagiaire. Je la remercie aussi de m’avoir encadré, pour sa disponibilité et surtout pour ses précieux conseils. Lui qui n’a jamais manqué de m’orienter et de me conseiller.

J’exprime aussi toute ma gratitude envers tout le personnel de Logilab qui ont consacré de leur temps pour m'aider à résoudre mes problèmes et pour avoir partagé leur connaissances avec moi.

Je remercie également mon tuteur de stage de m’avoir suivi pendant toute la période de stage.

\tableofcontents
\listoffigures
\listoftables

\chapter*{Introduction}
\addcontentsline{toc}{chapter}{Introduction}
J'ai effectué un stage d'une durée de quatre mois au sein du département Web Sémantique de Logilab : une société spécialisée dans le développement de solutions informatique pour la gestion de connaissances et d'applications scientifiques. Logilab développe et utilise CubicWeb, un framework de web sémantique distribué sous la licence libre \glslink{lgpl}{LGPL} écrit en langage Python. 

Ces dernières années les entreprises proposent de plus en plus leur services via des applications web (\textit{\glslink{saas}{Software As A Service}}) et utilisent l'informatique dans les nuages (\textit{cloud computing}) pour héberger ces applications. Les services en ligne doivent offrir des temps de réponse courts afin de satisfaire les utilisateurs. Le développement de ces services doit respecter les principes du web et principalement l'architecture \glslink{REST}{REST}. REST est un style d'architecture pour les systèmes hypermédia et est très adapté au World Wide Web. L'architecture REST définit des contraintes qui permettent, si elle sont respectées, d'avoir des applications web performantes, scalable et fiable.

Le sujet de mon stage porte sur la mise en place d'une architecture REST dans le framework CubicWeb. Dans un premier temps il s'agit de dresser l'état des lieux du respect des contraintes de cette l'architecture au sein du CubicWeb. Dans un deuxième temps refondre le code nécessaire pour respecter ces contraintes mais sans introduire de gros changement dans l'utilisation du CubicWeb. Enfin vérifier le fonctionnement des applications existantes et assurer qu'aucune régression n'a été introduite par le nouveau code.

J'ai choisi ce stage car le sujet s'inscrit parfaitement dans la logique de mon projet professionnel. En effet, mon objectif est de travailler dans le développement d'application d'Internet riche. Ces d'applications doivent avoir des temps de réponse courts. L'architecture REST est une solution pour cette contrainte. Le fait de travailler sur des projets libres m'a aussi motivé pour effectuer ce stage.

Ce rapport de stage détaille le travail effectué pour atteindre les objectifs fixé au début de stage. Je commencerai par présenter le contexte de stage : l'entreprise d'accueil, CubicWeb et l'architecture REST. Ensuite je détaillerai la méthode de travail utilisée durant la période de stage. Je présenterai ensuite le travail effectué pour atteindre les objectifs de stage. Enfin Je conclurai ce rapport par dresser le bilan des résultats obtenus.

\newmintedfile{xml}{frame=leftline,linenos}

\chapter{Contexte du stage}
\section{Logilab}
Logilab est une société de services d'une vingtaine de personnes créée en septembre 2000. Logilab a opté pour le modèle économique \textit{open source}. Logilab est spécialisée dans le développement de solutions informatiques principalement dans les domaines de gestion de connaissance et scientifique. Elle propose aussi du conseil et des formations couvrant de multiples sujets (Python, \glslink{xml}{XML}, conception orientée objet, C++, méthodes agiles, etc.).

Logilab contribue au Logiciel Libre et met plusieurs des ses développements sous la licence libre \glslink{lgpl}{LGPL}. Elle encourage ses employés à contribuer sur des projets libres et sponsorise plusieurs événements en relation avec le logiciel libre. Elle privilégie les solutions libres qui répondent aux besoins de l'utilisateur et offrent des garanties de stabilité. Elle privilégie Debian, la distribution de Linux non commerciale, qui est installée sur son parc informatique. Elle utilise majoritairement le langage Python pour le développement de ses outils. Logilab contribue à la communauté Python, co-organise la conférence annuelle EuroPython et a co-fondé Python Business Forum, une association européenne ayant pour objectif de promouvoir les utilisations de Python dans l'industrie.

Logilab est l'une des entreprises françaises expertes en langage Python et en web sémantique. En 2013, elle avait remporté le prix Stanford de l'innovation pour le projet Databnf réalisé pour le compte de la Bibliothèque nationale de France. Le même projet a aussi remporté le Trophée de l'Excellence « Data Intelligence », toutes catégories confondues, dans le cadre du salon Documation - MIS 2013\cite{dta}. Le projet a permet, en s’appuyant sur les technologies de web sémantique, d'exposer les catalogues de la Bibliothèque nationale de France. Dans le domaine scientifique, Logilab propose Simulagora, une plateforme de simulation de calcul scientifique dans les nuages.

Parmi les développement réalisé par Logilab et qui sont distribué sous licence libre on peut citer :
\begin{description}
\item[Pylint]\hfill\\ Un logiciel de vérification de code source et de la qualité du code pour le langage Python. Il utilise les recommandations officielles de style de la \glslink{pep}{PEP} ~8.
\item[CubicWeb]\hfill\\ Un framework web pour la réalisation d’applications web. Il supporte les standards du web sémantique.
\item[RQL]\hfill\\ \glslink{rql}{RQL} est Un langage de requête de haut niveau s’inspirant de langage SQL permettant d'interroger des sources de données.
\item[Mercurial] \hfill\\ plusieurs outils autour de logiciel de gestion de version Mercurial. Logilab propose Plusieurs extensions au logiciel et une application de revue de code pour les entrepôts Mercurial. 
\end{description}



\section{CubicWeb}
CubicWeb est un framework pour le développement d'applications du web sémantique écrit en langage Python. Son développement a commencé en 2001 et est utilisé pour l'intranet de Logilab. En octobre 2008, Logilab sort le framework sous la licence libre \glslink{lgpl}{LGPL}. En 2013, CubicWeb est lauréat du concours Dataconnexions 2013, organisé par Etalab\cite{etalab}.

Le framework permet à l'utilisateur de se concentrer sur le type de données manipulées par son application. Une fois le schéma de données est défini à l'aide de langage \glslink{yams}{YAMS}, CubicWeb génère des vues permettant de manipuler ces données avec différant types de formats (\glslink{html}{HTML}, \glslink{rdf}{RDF}, \glslink{json}{JSON}, \glslink{jsonld}{JSON-LD}, \glslink{xml}{XML}, etc.). CubicWeb supporte plusieurs sources de données comme les bases \glslink{sql}{SQL}, les annuaires \glslink{ldap}{LDAP}, les entrepôts de donnés Mercurial. Pour faciliter l'utilisation de CubicWeb dans le développement utilisant des méthodes agiles, CubicWeb fournit un outil pour réaliser la migration de l'application vers des versions plus récentes. Cet outil permet de changer le schéma de données à tout moment durant le développement de l'application.


Afin d'accélérer le développement d'une application, CubicWeb permet l'utilisation de composants appelés \emph{cubes}. Un cube est un composant CubicWeb fournissant une fonctionnalité. Plusieurs cubes peuvent être assemblés pour réaliser son application finale. Logilab avait développé un nombre important de cubes qui sont disponibles sous licence libre. Par exemple, une application web d'un blog peut utiliser le cube \textit{blog} qui fournit la fonctionnalité pour la rédaction de billets avec le cube \textit{comment} qui permet de rajouter des commentaires sur les billets. Toutes les fonctionnalités de l'application sont ainsi fournies par ces deux cubes. Il reste qu'à personnaliser les vues de ces cubes.

L'une des principales fonctionnalités de CubicWeb est son mécanisme de sélection des objets de l'application appelés \textit{appobjects} (vues, contrôleurs, services, etc.). Ces objets sont chargés au lancement de l'application et c'est le framework qui décide quel objet utiliser en fonction du contexte de la requête de l'utilisateur. Le framework se base sur le contexte pour attribuer des scores pour les \textit{appobjects} disponibles et pouvoir ainsi utiliser le meilleur adapté à la requête de l'utilisateur.  

\section{Pyramid-CubicWeb}
Un projet appelé Pyramid-Cubicweb est en développement depuis 2014 pour permettre de coupler CubicWeb avec le framework web Pyramid. Le but de ce projet est de remplacer Twisted pour servir les applications CubiccWeb. L'utilisation de Pyramid-CubicWeb permet aussi d'avoir accès aux fonctionnalités des deux framework. Logilab a commencé de migrer ces applications existantes pour utiliser Pyramid-CubicWeb.
 
\section{Architecture REST}
\begin{figure}
\centering
  \includegraphics[width=.6\textwidth]{tikz/rest.pdf}
  \caption{Un système respectant le style d'architecture REST}
  \label{fig:rest}
\end{figure}
\glslink{REST}{REST} (\textit{REpresentational State Transfer}) est un style d'architecture qui définit des contraintes pour les systèmes \glslink{hypermedia}{hypermédia} distribués. Il a été crée par Roy Fielding\footnote{L’un des principaux auteurs de la spécification \glslink{http}{HTTP} et membre fondateur de la fondation Apache} dans sa thèse de doctorat\cite{restthesise}. C'est le style adopté par plusieurs géant du web\footnote{comme Google, Facebook, Microsoft, .etc.} pour implanter leur services. 

REST définit six contraintes qui, si elle sont respectées, permettent d'avoir un système performant, scalable et fiable. Les six contraintes sont :

\begin{description}
\item[Client-Serveur]\hfill\\
Les responsabilités doivent être séparées entre le client et le serveur. Le serveur fournit une interface que le client peut utiliser. Cette contrainte permet aux deux d'évoluer indépendamment.

\item[Sans état]\hfill\\
C'est le client qui est responsable de stocker l'état de ses requêtes. La requête du client doit contenir toutes les informations nécessaires pour son exécution par le serveur. Cela permet d'avoir un système scalable. En effet en rajoutant un deuxième serveur, les requêtes des clients peuvent être partagés entre les deux serveurs d'une manière égale\footnote{Si l'état est stocké par le serveur, cet état doit être dupliqué sur chaque serveur.}.

\item[Mise en cache]\hfill\\
L'utilisation d'un cache pour ne pas renvoyer des données qui ont été déjà chargé par le client et n'ont pas changé. Cette contrainte permet d'améliorer les performances du système.

\item[Une interface uniforme]\hfill\\
Le serveur doit fournir au client une interface qui respecte les quatre règles suivantes :
\begin{itemize}
\item Chaque ressource doit avoir un identifiant
\item L'utilisation des représentations pour manipuler les ressources
\item Les requêtes et réponses sont auto-descriptifs
\item Hypermédia comme moteur d'état de l'application (\glslink{hateoas}{HATEOAS})	 
\end{itemize} 

\item[Un système hiérarchisé par couche]\hfill\\
Le système doit être hiérarchisé par couche et chaque couche a une responsabilité unique. Par exemple une couche peut avoir la responsabilité d'authentifier les clients et une autre peut offrir un cache. Cette contrainte permet l'évolution facile du système.

\item[Code à la demande]\hfill\\
Cette contrainte est optionnelle. Elle permet au serveur d'envoyer de code qui sera exécuté par le client. Cela permet d'améliorer les performances du système et permet au client d'évoluer au cours du temps.
\end{description}

En analysant ces contraintes, on trouve que le protocole \glslink{http}{HTTP} respecte bien le style d'architecture REST. Cela s'explique par le fait que Roy Fielding avait participé à l'écriture de ces spécifications et que REST est adapté au World Wide Web.  

Plusieurs services web REST sont apparus ces dernières années. Une grande partie des ces services ne respecte pas totalement l'architecture REST telle qu'elle est définie dans la thèse de de doctorat de Roy Fielding. C'est généralement la quatrième contrainte de l'interface uniforme qui n'est pas respectée. Normalement, lorsque cette contrainte est respectée, le client peut savoir tous les opérations qu'il peut effectuer sur les ressources. Ces opérations sont inclus dans la réponse de serveur. La figure ~\ref{fig:hateoas} montre un exemple incluant des liens permettant d'effectuer des opérations sur la ressource retournée. Dans l'exemple, la ressource retournée est un compte bancaire et on peut effectuer quatre opérations sur ce compte (dépôt, retrait, virement et fermeture). 


\begin{figure}[htp]
    \centering
    \RecustomVerbatimEnvironment{Verbatim}{BVerbatim}{}
        \xmlfile[firstline=1, lastline=13]{examples/examples.xml}
    \caption{Une réponse incluant les opérations possibles sur la ressource}
    \label{fig:hateoas}
\end{figure}




\chapter{Le stage}
Mon stage prote sur la mise en place d'une architecture REST dans le framework web CubicWeb. Les entreprises offrent de plus en plus des services web \glslink{REST}{REST}. Ces entreprises ont recours à ce type de services car ils sont facile à implémenter et à maintenir et permettent d'avoir de bonne performances. Même si l'architecture REST n'est pas standardisé, les développeurs des frameworks web essayent de respecter les contraintes définies par celle-ci.  
\section{Problématique}
Le développement de CubicWeb a commencé en 2000, la même année que REST a été crée. \`A cet époque là, REST n'était pas encore populaire. Les développeurs de CubicWeb n'ont pas pris l'architecture comme base pour le développement de celui-ci.  

Afin de pouvoir utiliser CubicWeb pour la réalisation d'applications 	performantes et qui pouvoir utiliser l'informatique dans nuages pour les héberger, celles-ci doivent respecter les principes fondateurs du web, principalement l'architecture REST.


\section{Objectifs}
Les objectifs de stage sont :
\begin{itemize}
	\item dresser l'état des lieux du respect des principes de l'architecture REST. Il s'agit de rédiger un document expliquant les points de l'architecture REST qui ne sont pas respectés par les applications CubicWeb.
	
	\item implémenter dans CubicWeb et dans Pyramid-CubicWeb des solutions permettant le respect de toutes les contraintes de l'architecture REST.  Les solutions implémentées doivent être compatibles avec les anciennes versions et ne doivent pas introduire de grand changement dans l'utilisation de CubicWeb.
	
	\item le dernier objectif est de vérifier le fonctionnement des applications existantes avec les nouvelles modifications et s'assurer qu'aucune régression n'a été introduite. 
\end{itemize}
\chapter{Démarche méthodologique}
\section{Organisation}
\subsection{Méthodes agiles}
Pour la réalisation des projets, Logilab laisse à ses clients le choix  entre une méthode de gestion de projet traditionnelle et une méthode agile mais Logilab privilégie cette dernière. La méthode agile utilisée par Logilab est très influencée par la méthode \glslink{methodexp}{XP}, \glslink{scrum}{Scrum} et \glslink{kanban}{Kanban}. En utilisant une méthode agile, un projet est développé d'une manière itératif et incrémental. Logilab et le client définissent les fonctionnalités à développer durant une itération. Ces fonctionnalités seront transformées en tâches (\textit{tickets}) et sont réalisées par ordre de priorité. La fin d'une itération donne lieu à un ensemble de livrables qui seront validés par le client. 

La méthode agile a été appliquée pour le développement du projet de stage. On avait utilisé principalement la méthode Kanban avec l'application de certaines pratiques de la méthode \glslink{xp}{XP} et de Scrum. Nous utilisons un tableau Kanban avec les colonnes suivantes : 
\begin{description}
	\item[backlog] l'ensemble des tâches à réaliser
	\item[ready] les tâches qui peuvent être réalisées et qui ne dépendent pas d'autres tâches non finies
	\item[doing] les tâches en cours de réalisation
	\item[review] les tâches en attente de revue.
\end{description}
\begin{figure}
\centering
  \includegraphics[width=\textwidth]{tikz/kanban.pdf}
  \caption{Cycle de vie d'une tâche avec la méthode Kanban}
  \label{fig:kanban}
\end{figure}
On commence par établir un ensemble de tâches sous forme de post-it qui sont ajoutées à la colonne \textit{backlog} du tableau Kanban. Les tâches qui peuvent être réalisées passent à la colonne \textit{ready}. La tâche la plus prioritaire passe à la colonne \textit{doing} du tableau. Une fois qu'une tâche est finie, elle passe à la colonne \textit{review} d'une autre personne. La personne ayant des tâches dans sa colonne \textit{review} s'occupe de faire la revue pour celles-ci et c'est lui qui décide si la tâche est finie ou pas. Si la tâche est considérée comme finie, le code développé sera intégré dans le projet, et dans le cas contraire la tâche retourne à la colonne \textit{ready} ou \textit{doing} de la personne travaillant sur la tâche. 

Des pratiques de la méthode XP sont appliquées tout au long du développement. Un document regroupant les bonnes pratiques et les conventions de codage est utilisé par Logilab et ces règles doivent être appliquées au code écrit. Le développement est piloté par les tests (\glslink{tdd}{TDD}), les tests sont développés avant les fonctionnalités. Cette pratique permet d'écrire un minimum de code qui fait passer les tests, ce qui minimise le temps de développement. Une fois les tests passent, le code passe à la phase de \textit{refactoring}, durant cette phase plusieurs opérations peuvent être faites dans le but d'améliorer la qualité de code (renommage, extraction des méthodes, utilisation des design pattern, etc.). \`A la fin de la phase de \textit{refactoring}, le code est envoyé à l’entrepôt distant. Une personne est choisi pour faire une première revue de code. La personne faisant la première revue peut demander la réalisation de tâches supplémentaires avant de le passer à phase d'intégration. 

\begin{figure}
\centering
  \includegraphics[width=.7\textwidth]{tikz/review.pdf}
  \caption{Le système de revue à Logilab}
  \label{fig:review}
\end{figure}

Tous les jours nous effectuions une réunion devant le tableau Kanban où chacun explique aux autres les tâches qu'il avait réalisé, les problèmes qu'il rencontre et les tâches sur lesquelles il va travaillé ensuite. Cette réunion permet de savoir l'avancement de tous les projets et permet le partage des idées sur les points bloquants. Chaque semaine, le personnel de Logilab répondent à un mail de RSH pour expliquer ce s'est bien ou mal passé durant la semaine précédente, le travail réalisé et le travail à faire durant la semaine. Une réunion rétrospective est organisée tous les mois durant laquelle nous discutant de ce qui a été fait durant le mois précédent et des problèmes rencontrés et nous proposons des solutions pour ceux-ci. 

\subsection{VSprint}
Le personnel de l'agence de Toulouse consacre la journée du vendredi de tous les quinze jours pour un VSprint. Durant le VSprint, tous le personnel développent ou améliorent les outils internes de l'entreprise. Chaque binôme se voit affecter un sujet par le directeur de l'agence. Le sujet est ensuite développé en \glslink{pairprogramming}{\textit{pair programming}}.

L'objectif du VSprint est de permettre le partage des connaissances et de bonnes pratiques de développement, de faire connaître les projets des autres développeurs et d'améliorer de façon continue les outils de l'entreprise. 

Une demie heure est aussi consacrée aux présentations (généralement deux). Des sujets sont sélectionnés et deux personnes se porte volontaires pour les présenter aux autres. Cette pratique est pertinente car c'est un moyen d'autoformation.

\section{Outils}
Plusieurs outils ont été utilisés pour la réalisation du projet de stage. 

\subsection{Python et JavaScript : langages de programmation}
Logilab est spécialisée dans le développement avec le langage de programmation Python. La grande partie des développement réalisée pendant le stage est alors écrite en Python. Python est un langage orienté objet, multi-paradigme qui est doté d'un typage dynamique. L'un des avantages de Python est la possibilité de modifier un programme sans avoir à changer directement le code de celui-ci (\textit{monkey patching}).

JavaScript est aussi utilisé pour l'écriture de code exécutant sur le navigateur de client. 

\subsection{Mercurial}
Mercurial est un logiciel de gestion de version décentralisé. Mercurial crée une branche pour chaque développeur d'un entrepôt. Il permet aussi d'associer un état (brouillon, publique, secret) à chaque nouvelle révision. Une nouvelle version commence dans l'état brouillon et passe à l'état publique lorsque elle est prête à être intégrer. 

Plusieurs extensions sont disponibles et qui rajoutent des fonctionnalités avancées au logiciel. Les extensions les plus utilisée durant le stage sont \textit{histedit} et \textit{evolve}. Ces deux extensions permettent de modifier les révisions dans le but d'avoir un historique propre. Elle permettent de changer les message d'une révision, fusionner des révisions et modifier une révision. Ces deux extensions garantissent qu'aucune donnée n'est perdu en modifiant l'historique. 

\subsection{Extranet}
Plusieurs outils sont disponibles sur l'extranet de l'entreprise. L'outil le plus utilisé est l'application de revue de code. \`A l'arrivé d'une nouvelle révision à l'entrepôt, l'application choisit une personne pour faire la revue. L'application permet aussi le visionnage des modifications apportées par cette révision. La révision est ensuite validée ou des tâches lui seront associées à réaliser avant sa validation. L'application permet aussi de voir à quelle tâche (\textit{ticket}) est attachée la nouvelle révision.   

\section{Planning}
Pour ce stage une méthode agile a été utilisé. Les méthodes agiles, contrairement à la méthode de gestion de projet, ne nécessite pas la réalisation d'un planning détaillé. La planification avec l'approche agile est adaptative et des ajustements peuvent être réalisés au cours de projet. Avec cette approche, le développement est réalisé de manière itératif et un macro-planning est réalisé et qui corresponde aux tâches nécessaires pour les développements à réaliser durant une itération. 

La figure ~\ref{fig:gantt} est le diagramme de Gantt qui correspond au planning réel de stage. On remarque que plusieurs tâches se déroulent simultanément et s'étale sur des périodes longues. Ceci est dû au processus de revue utilisé à Logilab. Lorsque une tâche est réalisée, celle-ci passe à la revue. Pendant que la tâche est en attente de revue, on commence à travailler sur une nouvelle tâche.

\newgeometry{margin=.7in}
\definecolor{barblue}{RGB}{153,204,254}
\definecolor{groupblue}{RGB}{51,102,254}
\definecolor{linkred}{RGB}{165,0,33}
\setganttlinklabel{s-s}{Début \`a d\'ebut (DD)}
\setganttlinklabel{f-s}{Fin \`a d\'ebut (FD)}
\uselanguage{French}
\languagepath{French}
\begin{landscape}
\begin{figure}
\centering
   \begin{ganttchart}[x unit=1.9mm, 
                      y unit chart=1.2cm, 
                      time slot format=isodate, 
                      %compress calendar,
                      vgrid,
                      %today=2014-05-17,
                      %today label=aujourd'hui,
                      %today label font=\scshape,
                      newline shortcut=true,
                      title/.append style={fill=blue!20},
                      title label font=\sffamily\bfseries\color{white},
                      title label node/.append style={below=-1.6ex},
                      title left shift=.05,
                      title right shift=-.05,
                      title height=1,
                      bar/.append style={draw=none, fill=black!63},
                      bar label node/.append style={align=center, font=\small},
                      bar incomplete/.append style={fill=barblue},
                      bar height=.4,
                      bar label font=\normalsize\color{black!50},
                      link/.style={-latex, draw=red, fill=red},
                      progress = today,
                      progress label text = {},
                     ]
                      {2015-05-04}{2015-08-28}
   \gantttitlecalendar{month=name} \\ 
   
    \ganttbar[progress=100]{Initiation\ganttalignnewline à CubicWeb}{2015-05-04}{2015-05-07} \\
    
    \ganttbar[progress=100]{\'Etat des lieux}{2015-05-10}{2015-05-14} \\
    
    \ganttbar[progress=100]{Rédaction\ganttalignnewline de la CWEP}{2015-05-17}{2015-05-31} \\
    
    \ganttbar[progress=100]{opération CRUD}{2015-06-01}{2015-08-28} \\
    
    \ganttbar[progress=100]{CubicWeb\ganttalignnewline sans état}{2015-06-08}{2015-06-24} \\
    
    \ganttbar[progress=100]{VSprint}{2015-06-17}{2015-06-22} \\
    
    
    \ganttbar[progress=100]{Négociation\ganttalignnewline de contenu}{2015-06-24}{2015-08-28} \\
    
    \ganttbar[progress=100]{VSprint}{2015-07-14}{2015-07-14} \\
    
    \ganttbar[progress=100]{Gestion de\ganttalignnewline requêtes RQL}{2015-07-17}{2015-07-21}\\
    
    \ganttbar[progress=100]{Rapport de\ganttalignnewline stage}{2015-08-10}{2015-08-21} \\
    
    \ganttlink[link type=f-s]{elem0}{elem1}
    \ganttlink[link type=f-s]{elem1}{elem2}
   
   \end{ganttchart}
\caption{Planning réel de stage}
\label{fig:gantt}
\end{figure}
\end{landscape}
\restoregeometry

\chapter{Conception et développement}
\section{État des lieux}
\label{edl}
Le premier objectif de mon projet de stage est de dresser l'état des lieux du respect des principes de l'architecture REST dans CubicWeb. Pour cela j'ai commencé par rédiger un document \glslink{cwep}{CWEP} (\textit{CubicWeb Enhancement Proposals}). J'ai commencé par réalisé une série de tests pour voir comment CubicWeb gère les requêtes web. J'ai d'abord crée une application CubicWeb utilisant le cube blog. J'ai lancé l'application et j'ai crée une entité qui sera utilisé pour les tests. Le tableau \ref{table:tests} illustre les résultats de ces tests. 

Les tests ont montré que CubicWeb ne respecte pas certaines spécification du protocole HTTP. En effet CubicWeb retourne des réponses non valides pour certaines requêtes HTTP. CubicWeb retourne des réponses ayant un format qui n'est pas géré par client. Les verbes HTTP n'ont pas de sémantique dans CubicWeb, en effet tous les verbes sont traité de la même manière. Pour avoir une réponse dans un format différent de \texttt{text/html}, un paramètre indiquant le format doit être inclut dans la requête. 

CubicWeb utilise la session pour stocker de données relatives à chaque requête HTTP. Lors de la modification d'une entité, les changement effectué sont stockés dans la session. Ceci permet de reprendre l'édition de l'entité même après avoir quitté la page d'édition. CubicWeb utilise un mécanisme appelé \textit{breadcrumbs} pour stocker les dix dernières URLs visitées par l'utilisateur.

Les tests ont montré que CubicWeb ne respecte pas les principes de l'architecture REST. CubicWeb ne respecte pas le deuxième principes car l'état des requêtes est stocké sur le serveur. L'application nécessite beaucoup de mémoire pour pouvoir stocker l'état de chaque requête. Ces données seront dupliquées sur chaque serveur hébergeant l'application. 

CubicWeb ne fournit pas une interface uniforme ce qui est contraire au quatrième principe de REST. Certaines URLs de CubicWeb ne respecte les principes d'une interface uniforme de REST. Il n'y a pas aussi une bonne utilisation des verbes HTTP. Le framework JavaScript de CubicWeb contient des méthodes faisant des requêtes utilisant un mauvais verbe HTTP (POST au lieu de GET). Cette mauvaise utilisation réduit l'efficacité de cache\footnote{Il n'est pas possible de mettre en cache les réponses pour des requêtes POST}.

\begin{table}[!h]
\begin{tabular}{|>{\raggedright\arraybackslash}m{.3\textwidth}|m{.3\textwidth}|m{.4\textwidth}|}
\hline
\multicolumn{1}{|>{\centering\arraybackslash}m{.3\textwidth}|}{\cellcolor{Gray}\textbf{Requête HTTP}} 
    & \multicolumn{1}{>{\centering\arraybackslash}m{.3\textwidth}|}{\cellcolor{Gray}\textbf{Réponse}} 
    & \multicolumn{1}{>{\centering\arraybackslash}m{.4\textwidth}|}{\cellcolor{Gray}\textbf{Commentaires}} \\
\hline
\tt{\footnotesize GET /blog/89\newline
	Accept:text/html} &
\tt{\footnotesize 200 OK \newline
	Content-Type:text/html} &
La réponse est valide
\\ \hline

\tt{\footnotesize GET /blog/89\newline
	Accept:application/json} &
\tt{\footnotesize 200 OK \newline
	Content-Type:text/html} &
La réponse n'est pas valide car elle n'est pas au bon format. La réponse doit être au format JSON si ce format est supporté par CubicWeb. S'il n'est supporté une réponse \tt{406~Not~Acceptable} devait être retourné.  
\\ \hline

\tt{\footnotesize DELETE /blog/89} &
\tt{\footnotesize 200 OK \newline
	Content-Type:text/html} &
La requête est traité comme une requête GET et le billet n'a pas été supprimé. CubicWeb devait retourné une réponse \tt{405~Method Not Allowed}.
\\ \hline 

\tt{\footnotesize PUT /blog/89} &
\tt{\footnotesize 200 OK \newline
	Content-Type:text/html} &
La requête est traité comme une requête GET et le billet n'a pas été mis à jour. CubicWeb devait retourné une réponse \tt{405~Method Not Allowed}.
\\ \hline 

\tt{\footnotesize POST /blog} &
\tt{\footnotesize 200 OK \newline
	Content-Type:text/html} &
La requête est traité comme une requête GET et le billet n'a pas été crée. CubicWeb devait retourné une réponse \tt{405~Method Not Allowed}.
\\ \hline 
\end{tabular}
\caption{Résumé des tests}
\label{table:tests}
\end{table}

\section{CubicWeb sans état}
Comme a été expliqué dans la section \ref{edl}, CubicWeb utilise la session pour stocker l'état d'une requête pour pouvoir exécuter les prochaines requêtes de même client. Cette pratique ne respecte pas les principes de REST. Stocker des données rajoute sur les serveur nécessite des traitements supplémentaire and et réduit les performances. Implanter des solutions pour transférer l'état des requêtes vers le client était la tâche la plus prioritaire.

Pour concevoir une solution pour ce problème j'ai commencé par déterminer le type de données qui sont stocker par serveur. CubicWeb stocke les dix dernières pages visitées par le client dans le but de savoir vers quelle page se rediriger lorsque le client annule une action dans une page donnée. CubicWeb stocke aussi les modifications effectuées sur une entité lors de son édition. Lorsque l'utilisateur valide les modifications, celles-ci sont appliquées à l'entité et la session est vidée. Ceci est dans le but de permettre à l'utilisateur de reprendre l'édition d'une entité même après avoir fermé la page d'édition. 

Pour retrouvé la page vers laquelle se rediriger lorsque l'utilisateur annule une action, j'ai implanté une solution en JavaScript. JavaScript permet de naviguer dans l'historique de navigateur. Pour informer le serveur de l'annulation de l'action, une requête \glslink{ajax}{Ajax} est envoyée au serveur en utilisant les librairies JavaScript de CubicWeb. Cette solution est exécuté par le client ce qui réduit la charge du serveur. Cette solution a permet aussi de nettoyer une partie de code dans CubicWeb.

Pour ne pas se servir de la session pour l'édition des entités, j'ai implanter une solution qui permet de passer les modifications dans les données de la requête validant l'édition. Avec cette solution l'utilisateur effectue ces modifications et celles-ci sont stockés dans le client. Cette solution permet de réduire le nombre de requêtes effectuées par le client lors d'édition d'une entité. En effet avec l'ancienne solution, une requête est envoyée pour chaque modification. Cette solution ne permet pas de reprendre une édition si la page d'édition a été fermé. Pour cette raison, cette solution n'a pas été retenue.

\section{Opération CRUD avec Pyramid}

\section{Négociation de contenu} 
 


\chapter{Bilan}
\section{Bilan des résultats}
Le travaille effectué pendant ce stage a permis de dresser l'état des lieux du respect des contraintes de l'architecture \glslink{REST}{REST} dans CubicWeb. Un document \glslink{cwep}{CWEP} a été rédigé qui détaillent les points de CubicWeb ne respectant pas REST. Le document classe les applications CubicWeb selon le modèle développé par Leonard Richardson\cite{rmm}. Ce modèle permet de classer une application selon quatre niveaux (de 0 à 3) et définit comment passer d'un niveau à un autre. Le niveau 3 est une condition nécessaire pour réaliser une application respectant complètement l'architecture REST telle qu'elle est définie par Roy Fielding. Je n'ai pas eu beaucoup de retours sur le contenu de ce document, certains points reste des questions ouvertes. Les documents CWEP sont utilisés par les développeurs de CubicWeb pour implanter de nouvelles évolutions de framework. Ces points vont être alors abordés dans l'avenir et des solutions peuvent être implantés pour les corriger. 

J'ai pu proposer des solutions pour certains points. J'ai implanté une solution qui permettait de supprimer le mécanisme de \textit{breadcrumbs} utilisé dans CubicWeb. Cette solution permet de réduire la quantité de donnés stockée sur le serveur et le nombre de traitements effectués par celui-ci. J'ai aussi implanté dans Pyramid-CubicWeb des vues permettant de faire les opérations \glslink{crud}{CRUD} en respectant REST. La solution fonctionne dans CubicWeb lorsque celui-ci est utilisé avec Pyramid-CubicWeb. 

Enfin j'ai implanté la négociation de contenu qui va permettre au client d'une application CubicWeb de choisir le format de données à utiliser via l'entête HTTP \texttt{Accept}. Cette solution respecte bien les spécification de protocole HTTP et ainsi les principes de REST. 

Durant le stage, j'ai participé aux VSprint et j'ai contribué à améliorer les outils interne de Logilab. J'ai développé une solution permettant d'avoir la différence entre la dernière version d'une révision et une version obsolète.     
\subsection{Atteinte des objectifs}
Dans l'ensemble les objectifs qui ont été fixés au début de stage ont été atteints. Le premier objectif de stage qui est de dresser l'état des lieux du respect des principes REST dans CubicWeb a été atteint. Un document détaillant l'état des lieux a été réalisé et peut être utilisé pour faire évoluer CubicWeb. 

Les deuxième objectif qui est de refondre dans CubicWeb le code nécessaire pour respecter les principes de REST n'est pas complètement atteint. La solution intégrée ne permet pas de transférer complètement l'état des requêtes vers le client. CubicWeb stocke toujours des données relatives à chaque requête. La solution qui permet de transférer complètement l'état vers le client introduit un grand changement du l'utilisation de l'interface web de Cubicweb. 

Le troisième objectif a été atteint puisque les applications existantes fonctionne encore avec le code écrit.

\subsection{Perspective d'évolution}
La solution proposée pour réaliser des opérations \glslink{crud}{CRUD} sur les entités peut être améliorer et respecter les contraintes de niveau 3 du modèle de Leonard Richardson. L'évolution possible est de rajouter des contrôles Hypermédia au réponses retournées. Le principe des contrôles Hypermédia est d'inclure des URLs pour indiquer les opérations qui peuvent être faites sur la ressource retournée. 

\section{Bilan personnel}
Ce stage est très enrichissant sur plan personnel. Il m'a permis de mettre en application les connaissances que j'ai acquises durant mon cursus universitaire. Il m'a aussi permis d'acquérir de nouvelles connaissances techniques. Ce stage était l'occasion pour moi de contribuer à un projet sous licence libre. Grâce à cette expérience j'ai acquis les bonnes pratiques pour contribuer de manière efficace aux projets libres.  
\subsection{Connaissances acquises}
Ce stage est aussi enrichissant en terme de connaissances acquises. J'ai découvert deux framework web : CubicWeb et Pyramid. CubicWeb étant un framework de web sémantique m'a permis d'avoir une ouverture vers ce domaine. 

Ce stage était l'occasion pour moi d'améliorer mes connaissances des langages Python et JavaScript. J'ai pu découvrir des fonctionnalités avancés de ces deux langages. J'ai appris à utiliser la librairie JavaScript JQuery qui est très utilisée dans le développement web. 

Durant le stage j'ai appris à utiliser le gestionnaire de version Mercurial. J'ai été amené à utiliser les fonctionnalité de réécriture de l'historique de celui-ci. En effet, il était indispensable de modifier les révisions envoyées sur l'entrepôt dans le but d'avoir un historique propre. Cette année, j'ai suivi les cours sur l'utilisation de Git mais les cours ne parlait pas de cette fonctionnalité que je trouve très indispensable. 

J'ai appris à utiliser plusieurs outils comme l'éditeur VIM et j'ai découvert Salt, un outil de gestion de configuration. 
\subsection{Projet professionnel}
J'ai décidé de faire ce stage car il s'inscrit parfaitement dans la logique de mon projet professionnel qui est de travailler dans le domaine de web pour la réalisation d'applications d'Internet riches. Ce stage m'as permet d'acquérir de bonnes connaissances dans ce domaine.  


\chapter*{Conclusion} 
\addcontentsline{toc}{chapter}{Conclusion} 
Ce stage que j'ai effectué à Logilab a permis de montrer les différentes 
implantations de CubicWeb ne respectant pas les principes de l'architecture 
REST. Un document a été réalisé détaillant ces points et liste plusieurs 
évolutions possibles de CubicWeb. J'ai eu l'occasion de corriger certains de 
ces point mais il reste beaucoup de travail à faire pour mettre en place une 
architecture REST dans CubicWeb.     

Ce stage m'a permis de découvrir comment sont appliquées les méthodes agiles
pour les projets informatiques dans l'industrie. J'ai vu aussi comment il est
possible d'adapter plusieurs méthodes agiles pour les besoins de l'entreprise.

Ce stage est très enrichissant sur le plan techniques. Cela s'explique par le
fait que Logilab est très experte dans plusieurs domaine et met en place un
système efficace permettant le partage des connaissances entre l'ensemble de
ses employés. J'ai eu l'occasion d'approfondir mes connaissances sur
l'architecture REST. Durant ce stage j'ai étudié REST en partant de zéro et en
utilisant comme base la thèse de Roy Fielding le créateur de REST. Ceci m'a
permis de bien comprendre les contraintes définies dans REST. 

J'ai aussi amélioré mes connaissances des langages Python et JavaScript et
durant ce stage j'ai pu découvrir les fonctionnalités avancés de ces deux
langages. 

Pour conclure, ce stage était une expérience très enrichissante qui va me
servir dans la suite de mes études mais aussi dans le cadre professionnel.




%%%%%%%%%%%%%%%%%%%%%%%%%%%%%%
\newglossaryentry{scrum}
{
	name={Scrum},
	description={Scrum est une méthode agile créée en 2001 par Jeff Sutherland et Ken Schwaber qui s'appuie sur le découpage d'un projet en plusieurs itérations appelées sprint d'une durée qui varie de deux semaine à un mois. Le processus défini par cette méthode s'articule autour d'une équipe unie cherchant à atteindre un but précis},
}

\newglossaryentry{methodexp}
{
	name={M\'ethode XP},
	description={La méthode XP ou \textit{eXtreme Programming} est une méthode agile de gestion de projet informatiques. Contrairement à Scrum, la méthode XP définit des pratiques pour la réalisation de produit. Comme ces pratiques sont liées au génie logiciel, cette méthode ne peut être utilisée que pour les projets informatiques },
}

\newglossaryentry{kanban}
{
	name={Kanban},
	description={La méthode Kanban permet, à l'aide d'un tableau, de visualiser le processus de traitement d'une tâche. Les colonnes du tableau sont les états possibles pour la tâche et chaque colonne contient un nombre maximum prédéfini de tâches simultanées. Les tâches passent alors d'un état à un autre selon l'avancement de projet}
}

\newglossaryentry{pairprogramming}
{
	name={Programmation en binôme},
	description={C'est une méthode de travail dans laquelle deux développeurs travaillent ensemble sur un poste de travail. Une personne rédige le code et une autre observe pour déceler des problèmes dans le code écrit par l'autre personne. C'est l'une pratiques de la méthode XP}
}

\newglossaryentry{hypermedia}
{
	name={Hyperm\'edia},
	description={Hypermédia est une extension de l'hypertexte à des données multimédias comme des images, sons et vidéos.}
}

\newglossaryentry{hateoas}
{
	name={HATEOAS},
	description={\textit{Hypermedia as the Engine of Application State} (Hypermédia comme moteur d'état de l'application) est une contrainte de l'architecture REST qui la distingue des autres architectures. Le principe c'est que le client interagie avec une application en utilisant des hypermédia inclus dans la réponse de l'application. Ceci permet de faire évoluer l'interface fournie par le serveur sans changer le client}  
}

\newglossaryentry{monkeypatching}
{
	name={Monkey patching},
	description={\textit{Monkey patching} est une façon de modifier ou d'étendre du code sans modifier le code source original. Ceci est possible avec les langages de programmations dynamiques.}  
}


\printglossaries


%%%%%%%%%%%%%%%%%%%%%%%%%%%%%%
\nocite{*}
\cleardoublepage
\phantomsection
\addcontentsline{toc}{chapter}{Bibliographie}
\printbibliography
\end{document}
