\selectlanguage{french} 
\begin{abstractpage} 

\begin{abstract}{french} 
Dans le cadre de mon Master 1 Développement Logiciel, j'ai eu l’opportunité
d'effectuer un stage d'une durée de 4 mois au sein de la société Logilab qui
est spécialisée dans le développement web sémantique et scientifique.
    
Logilab participe au développement du Logiciel Libre et met à disposition
certains de ses développements sous licence libre. Elle développe CubicWeb,
un framework pour le web sémantique écrit en langage Python qui facilite le
développement d'applications web à partir du schéma du modèle de données qu'elle manipule 
et grâce à la réutilisation des composants appelés \textit{cubes}.

Ces dernières années il y eu a une multiplication des services en ligne
(\textit{\glslink{saas}{Software As A Service}}), ces derniers ayant de plus en plus recours à
l'informatique dans les nuages (\textit{cloud computing}). Pour supporter la forte charge et les
besoins en disponibilité sur ces services distribués, le style d'architecture \glslink{REST}{REST}
(\textit{Representational State Transfer}) a ressurgi. REST a été théorisé par Roy Fielding à la
naissance du World Wide Web alors qu'il concevait le protocol HTTP.  C'est un style architecture
pour les systèmes hypermédia distribués, dont HTTP est une implémentation. En suivant les principes
qu'il a énoncés, on peut donc utiliser ce dernier et tirer pleinement parti de l'architecture du Web
pour obtenir des sites et services supportant une très forte charge avec une haute disponibilité.

Mon stage avait comme objectif de dresser l'état des lieux du respect des
principes REST dans CubicWeb, modifier le code de celui-ci pour respecter ces
principes et vérifier le fonctionnement des applications existantes avec les
nouvelles modifications.  \end{abstract} {\textbf{\textit{Mots clés---}}
CubicWeb, REST, Web sémantique, cubes} 

\newpage 


\begin{abstract}{english} 
As part of my Master 1 Software Development, I did my four months internship
at Logilab : a company specialized in semantic web and scientific development. 
Logilab participates to the Free Software and releases some of its developments 
under a free software license. It develops CubicWeb, a semantic web framework written in 
Python that ease the development of a web applications from its data model schema and 
by reusing components called \textit{cubes}.

These last years there has been a growing number of online services
(\glslink{saas}{Software As A Service}) and the use of cloud computing. It then
became necessary that web applications respect the principles of the web
development, but mainly the REST architecture (Representational State Transfer).
REST is an architecture style for distributed hypermedia systems and well suited
for the World Wide Web.

The purpose of my internship is to make an inventory of CubicWeb's compliance
with \glslink{REST}{REST} principles, change the code  to make it respect these
principles and make sure that existing applications still work with the new
changes.  \end{abstract} {\textbf{\textit{Keywords---}} CubicWeb, REST, Semantic
web, cubes} \end{abstractpage}

\selectlanguage{french}
