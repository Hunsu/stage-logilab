\newmintedfile{python}{frame=leftline,linenos}
\newcolumntype{C}[1]{>{\centering}m{#1}
}
\chapter{Conception et développement}
\section{État des lieux}
\label{edl} 
Le premier objectif de mon projet de stage est de dresser l'état
des lieux du respect des principes de l'architecture REST dans CubicWeb. Pour
cela j'ai commencé par rédiger un document \glslink{cwep}{CWEP}
(\textit{CubicWeb Evolution Proposals}) : un document pour proposer des
évolutions pour CubicWeb. J'ai commencé par réaliser une série de tests pour
voir comment CubicWeb gère les requêtes web. J'ai d'abord crée une application
CubicWeb utilisant le cube blog. J'ai lancé l'application et j'ai crée une
entité qui sera utilisée pour les tests. Le tableau ~\ref{table:tests} illustre
les résultats de ces tests. 

Les tests ont montré que CubicWeb ne respecte pas certaines spécification du
protocole \glslink{http}{HTTP}. En effet, CubicWeb retourne des réponses non
valides pour certaines requêtes HTTP. Par exemple, CubicWeb retourne des
réponses ayant un format qui n'est pas géré par client. Les verbes HTTP n'ont
pas de sémantique dans CubicWeb, tous les verbes sont traités de la même
manière. Il ne fournit pas un moyen pour faire la négociation de contenu :
l'en-tête \texttt{Accept} est ignoré par celui-ci.  

CubicWeb utilise la session pour stocker des données relatives à chaque requête
HTTP. Lors de la modification d'une entité, les changement effectuées sont
d'abord stockés dans la session. Ceci permet de reprendre l'édition de l'entité
même après avoir quitté la page d'édition. CubicWeb utilise un mécanisme appelé
\textit{breadcrumbs} pour stocker les dix dernières \glslink{url}{URLs}
visitées par l'utilisateur.

Les tests ont montré que CubicWeb ne respecte pas les principes de
l'architecture REST. En effet, il ne respecte pas le deuxième principe car
l'état des requêtes est stocké sur le serveur. L'application nécessite plus de
mémoire pour pouvoir stocker l'état de chaque requête. Si plusieurs serveurs
sont utilisés pour héberger l'application, ces données doivent être dupliquées
sur chaque serveur pour avoir une répartition égale de la charge. 

CubicWeb ne fournit pas une interface uniforme ce qui est contraire au
quatrième principe de REST. Certaines de ses URLs ne respectent pas les
principes d'une interface uniforme de REST. Les verbes HTTP ne sont pas
utilisés correctement. Les librairies JavaScript de CubicWeb contiennent des
fonctions faisant des requêtes utilisant un mauvais verbe HTTP (POST au lieu de
GET). Cette mauvaise utilisation réduit l'efficacité de cache\footnote{Il n'est
pas possible de mettre en cache les réponses pour des requêtes POST}.

\begin{table}[!h]
    \begin{tabular}{|>{\raggedright\arraybackslash}m{.3\textwidth}|m{.3\textwidth}|m{.4\textwidth}|}
        \hline
        \multicolumn{1}{|>{\centering\arraybackslash}m{.3\textwidth}|}{\cellcolor{Gray}\textbf{Requête
        HTTP}} &
                \multicolumn{1}{>{\centering\arraybackslash}m{.3\textwidth}|}{\cellcolor{Gray}\textbf{Réponse}}
                &
                \multicolumn{1}{>{\centering\arraybackslash}m{.4\textwidth}|}{\cellcolor{Gray}\textbf{Commentaires}}
                \\ \hline \tt{\footnotesize GET /blog/89\newline
                Accept:text/html} & \tt{\footnotesize 200 OK \newline
                Content-Type:text/html} & La réponse est valide.  \\ \hline

\tt{\footnotesize GET /blog/89\newline Accept:application/json} &
                    \tt{\footnotesize 200 OK \newline Content-Type:text/html} &
                    La réponse n'est pas valide car elle n'est pas au bon
                    format. La réponse doit être au format JSON si ce format
                    est supporté par CubicWeb. Dans le cas contraire, une
                    réponse \texttt{406~Not~Acceptable} devait être retournée.
                    \\ \hline

\tt{\footnotesize DELETE /blog/89} & \tt{\footnotesize 200 OK \newline
                Content-Type:text/html} & La requête est traitée comme une
                    requête GET et le billet n'a pas été supprimé. CubicWeb
                    devrait retourner une réponse \tt{405~Method Not Allowed}.
                    \\ \hline 

\tt{\footnotesize PUT /blog/89} & \tt{\footnotesize 200 OK \newline
                Content-Type:text/html} & La requête est traitée comme une
                    requête GET et le billet n'a pas été mis à jour. CubicWeb
                    devrait retourner une réponse \tt{405~Method Not Allowed}.
                    \\ \hline 

\tt{\footnotesize POST /blog} & \tt{\footnotesize 200 OK \newline
                Content-Type:text/html} & La requête est traitée comme une
                    requête GET et le billet n'a pas été crée. CubicWeb devrait
                    retourner une réponse \tt{405~Method Not Allowed}.  \\
                \hline \end{tabular} \caption{Résumé des tests}
            \label{table:tests} \end{table}

\section{CubicWeb sans état} 
Comme a été expliqué dans la section \ref{edl}, CubicWeb utilise la session pour 
stocker l'état d'une requête pour pouvoir exécuter les prochaines requêtes de même 
client. Cette pratique ne respecte pas les principes de REST. Stocker des données 
sur le serveur nécessite des traitements supplémentaires ce qui réduit les 
performances. Implanter des solutions pour transférer l'état des requêtes vers le 
client était la tâche la plus prioritaire.

Pour concevoir une solution pour ce problème j'ai commencé par déterminer le
type de données qui sont stockées sur le serveur. CubicWeb stocke les dix
dernières \glslink{url}{URLs} visitées par le client dans le but de savoir vers
quelle page se rediriger lorsque le client annule une action dans une page
donnée. CubicWeb stocke aussi les modifications effectuées sur une entité lors
de son édition. Lorsque l'utilisateur valide les modifications, celles-ci sont
appliquées à l'entité et sont enlevées de la session. Ceci est dans le but de
permettre à l'utilisateur de reprendre l'édition d'une entité même après avoir
fermé la page d'édition et aussi pour permettre le rajout des relations. 

Pour retrouver la page vers laquelle se rediriger lorsque l'utilisateur annule
une action, j'ai implanté une solution en JavaScript. JavaScript permet de
naviguer dans l'historique de navigateur et retrouver ainsi la page précédente.
Pour informer le serveur de l'annulation de l'action, une requête
\glslink{ajax}{Ajax} est envoyée au serveur en utilisant les librairies
JavaScript de CubicWeb. Cette solution est exécutée par le client ce qui réduit
la charge du serveur. Cette solution a permet aussi de nettoyer une partie de
code dans CubicWeb qui gérait le mécanisme de \textit{breadcrumbs}.

Pour ne pas se servir de la session pour l'édition des entités, j'ai implanté
une solution qui permet d'envoyer les modifications dans le corps de la requête
validant l'édition. Avec cette solution l'utilisateur effectue ses
modifications et celles-ci sont stockées dans le client. Cette solution permet
de réduire le nombre de requêtes effectuées par le client lors d'édition d'une
entité. En effet, avec l'ancienne solution, une requête est envoyée pour chaque
modification. Cette solution ne permet pas de reprendre une édition si la page
d'édition a été fermée. Pour cette raison, cette solution n'a pas été retenue.

\section{Opérations CRUD avec Pyramid} Les frameworks web fournissent
généralement des vues permettant de manipuler les données de l'application web.
Ils fournissent les quatre opérations suivantes appelées \glslink{crud}{CRUD} :

\begin{description} 
    \item[Create] créer une nouvelle donnée 
    \item[Read] retrouver une donnée 
    \item[Update] mettre à jour une donnée 
    \item[Delete] supprimer une donnée 
\end{description}

CubicWeb génère des vues HTML permettant de faire des opérations CRUD sur les
entités. Certaines de ces opérations ne sont pas implantées d'une manière
respectant les principes de REST. Le protocole HTTP définit une méthode pour
chaque opération (tableau ~\ref{table:crud}) et ceci n'est pas respecté par
CubicWeb. En effet, l'opération permettant de retrouver une entité utilise bien
la bonne méthode mais pas les trois autres. Aussi, les requêtes ne sont pas
envoyées à la bonne \glslink{url}{URL}.

On a décidé d'implanter ces opérations dans le projet Pyramid-CubicWeb afin
de pouvoir utiliser les fonctionnalités fournies par le framework Pyramid.
Pyramid facilite la création des vues et gère les requêtes HTTP. Créer une vue
Pyramid revient à écrire une seule fonction qui a accès à toutes les
informations de la requête. Pyramid permet l'utilisation des décorateurs Python
pour la configuration de la vue. Il permet de spécifier son URL, la méthode
HTTP, le format de données qu'elle renvoie, etc. Pyramid utilise la
configuration de chaque vue pour réaliser le routage des requêtes vers les
bonnes vues. 

La figure ~\ref{fig:delete} est le code de la vue créée pour l'opération de
suppression\footnote{seulement 10 lignes de code}. La vue supprime l'entité qui
est récupérée via le contexte de requête et renvoie une réponse HTTP de type
\texttt{204~No~Content} indiquant le succès de la requête. L'URL de cette vue
est \url{/etype/eid} qui est l'URL REST d'une entité qui définit son type et
son identifiant. Cette URL est rajoutée dans la configuration de Pyramid
(Figure ~\ref{fig:figpyramidconf}).

    \begin{table} \centering
        \begin{tabular}{|>{\centering\arraybackslash}m{.3\textwidth}
            |>{\centering\arraybackslash}m{.3\textwidth}
        |>{\centering\arraybackslash}m{.4\textwidth}|} \hline \cellcolor{Gray}
        \textbf{Opération} & \cellcolor{Gray} \textbf{Méthode HTTP} &
        \cellcolor{Gray} \textbf{URL} \\ \hline Create & POST & L'URL de type
        de la donnée\\ \hline Read   & GET & L'URL de la donnée\\ \hline Update
                                      & PUT ou PATCH & L'URL de la donnée \\
        \hline Delete & DELETE & L'URL de la donnée \\ \hline \end{tabular}
    \caption{Correspondance entre les opérations CRUD et les méthodes HTTP}
\label{table:crud} \end{table}


\begin{figure}[htp] 
    \centering
    \RecustomVerbatimEnvironment{Verbatim}{BVerbatim}{} 
    \pythonfile[firstline=35,lastline=45]{examples/examples.py} 
    \caption{Une vue Pyramid pour l'opération de suppression} 
    \label{fig:delete} 
\end{figure}

\begin{figure}[htp] 
    \centering
    \RecustomVerbatimEnvironment{Verbatim}{BVerbatim}{} 
    \pythonfile[firstline=48,lastline=55]{examples/examples.py} 
    \caption{Exemple de configuration avec Pyramid} 
    \label{fig:figpyramidconf} 
\end{figure}



Après avoir implanté ces vues, il fallait modifier CubicWeb pour les utiliser. 
On ne voulait pas modifier directement le code de CubicWeb car
les vues sont dans le projet Pyramid-CubicWeb et celui-ci n'est pas toujours
utilisé avec CubicWeb. Il a fallu modifier CubicWeb seulement s'il est utilisé
avec Pyramid-CubicWeb. J'ai crée alors des
\glslink{monkeypatching}{\textit{monkey patchs}} pour réaliser cette tâche. Le
code de ces \textit{monkey patchs} est écrit dans Pyramid-CubicWeb et modifie
CubicWeb au moment de l'exécution. Si Pyramid-CubicWeb n'est pas utilisé, le
code n'est pas exécuté et aucune modification ne sera faite.

\section{Négociation de contenu} 
L'interface uniforme de l'architecture REST spécifie les représentations comme 
un moyen pour manipuler les ressources. Une représentation est le format utilisé 
pour représenter la ressource. Le serveur peut proposer plusieurs formats pour ses 
ressources et c'est au client de choisir lequel utiliser. Le protocole HTTP définit 
un moyen pour faire la négociation de contenu entre le client et le serveur. Le 
client spécifie dans l'en-tête \texttt{Accept} les représentations qu'il souhaite 
recevoir par ordre de priorité. Le serveur envoie la ressource dans la 
représentation la mieux adaptée au client. Si le serveur ne peut pas retourner la 
ressource en aucune des représentations demandées par le client, une réponse de type
\texttt{406~Not~Acceptable} doit être retournée.

CubicWeb ne prend pas en compte l'en-tête \texttt{Accept} lors de la sélection
des vues. CubicWeb peut renvoyer une ressource au format HTML même si le client
a demandé la représentation \glslink{json}{JSON} de celle-ci. Le seul moyen
pour forcer CubicWeb à retourner un format spécifique est de spécifier
l'identifiant de la vue à utiliser. Chaque vue CubicWeb dispose d'un
identifiant et peut être appelé en rajoutant à la requête un paramètre
\texttt{vid} ayant comme valeur l'identifiant de la vue. Deux solutions ont été
implantées pour des projets client mais n'ont pas été intégrées dans CubicWeb.
L'une des solutions utilise le mécanisme de sélection de CubicWeb. Le mécanisme
de sélection fonctionne avec des prédicats. Chaque vue est associée à un
prédicat et lors de la sélection, les prédicat sont exécutés et chaque vue se
voit affecter un score. La vue ayant le score le plus haut est alors
sélectionnée et son contenu est retourné. Cette solution fonctionne mais ne
retourne pas la bonne réponse dans le cas où le format demandé par le client
n'est pas supporté. L'autre solution implante la négociation de contenu dans
une vue. Cette vue est toujours sélectionnée et c'est elle qui décide selon
l'en-tête \texttt{Accept} vers quelle vue se rediriger. Cette solution ne
retourne pas non plus une bonne réponse en cas d'échec de négociation.

La négociation de contenu est disponible dans Pyramid. Pyramid implante la
fonctionnalité de négociation de contenu comme ça été spécifié par le protocole
HTTP. Les vues Pyramid peuvent être configurées pour indiquer le format qu'elle
retourne. Pyramid utilise ces informations durant la phase de négociation de
contenu pour sélectionner la vue à utiliser. J'ai alors modifié la
configuration des vues crées pour réaliser les opérations CRUD. Pour les vues
CubicWeb nous avons décidé de s'appuyer sur une convention pour choisir les
identifiants des vues. Les vues qui retournent de contenu de type
\texttt{text/html} auront \texttt{primary} comme identifiant. Les autres vues
auront l'identifiant \texttt{primary.format}, une vue qui retourne de contenu
au format \glslink{json}{JSON} aura l'identifiant \texttt{primary.json}. J'ai
crée alors des vues pour les format pris en charge par CubicWeb et dans ces
vues on sélectionne une vue parmi les vues supportant le format de client.   


\begin{figure}[htp] 
    \centering
    \RecustomVerbatimEnvironment{Verbatim}{BVerbatim}{} 
    \pythonfile[firstline=58,lastline=79]{examples/examples.py} 
    \caption{Une vue Pyramid retournant des entities au format JSON} 
    \label{fig:getjson} 
\end{figure}
