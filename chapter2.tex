\chapter{Le stage} 
Mon stage porte sur la mise en place d'une architecture REST dans le framework 
web CubicWeb. Les entreprises offrent de plus en plus des services web 
\glslink{REST}{REST}. Ces entreprises ont recours à ce type de services car ils 
sont facile à implanter et à maintenir et permettent d'avoir de bonnes 
performances. Même si l'architecture REST n'est pas standardisé, les développeurs 
des frameworks web essayent de respecter les contraintes définies par celle-ci.   

\section{Problématique} Le développement de CubicWeb a commencé en 2000. \`A cet 
époque là, REST n'était pas encore populaire. Les développeurs de CubicWeb n'ont 
pas pris l'architecture comme base pour le développement de celui-ci.  

Afin de pouvoir utiliser CubicWeb pour la réalisation d'applications performantes 
et qui pouvoir utiliser l'informatique dans nuages pour les héberger, celles-ci 
doivent être réalisées en respectant les principes fondateurs du web, principalement 
l'architecture REST.


\section{Objectifs} 
Les objectifs de stage sont : 

\begin{itemize} 
    \item dresser l'état des lieux du respect des principes de l'architecture REST.
        Il s'agit de rédiger un document détaillant les points de l'architecture REST 
        qui ne sont pas respectés par les applications CubicWeb et comment mettre en 
        place cette architecture dans celui-ci.
	
    \item implanter dans CubicWeb et dans Pyramid-CubicWeb des solutions
        permettant le respect de toutes les contraintes de l'architecture REST. 
	
    \item le dernier objectif est de vérifier le fonctionnement des applications 
        existantes avec les nouvelles modifications et s'assurer qu'aucune régression 
        n'a été introduite.  
\end{itemize}
