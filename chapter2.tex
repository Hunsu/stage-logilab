\chapter{Le stage}
Mon stage prote sur la mise en place d'une architecture REST dans le framework web CubicWeb. Les entreprises offrent de plus en plus des services web \glslink{REST}{REST}. Ces entreprises ont recours à ce type de services car ils sont facile à implémenter et à maintenir et permettent d'avoir de bonne performances. Même si l'architecture REST n'est pas standardisé, les développeurs des frameworks web essayent de respecter les contraintes définies par celle-ci.  
\section{Problématique}
Le développement de CubicWeb a commencé en 2000, la même année que REST a été crée. \`A cet époque là, REST n'était pas encore populaire. Les développeurs de CubicWeb n'ont pas pris l'architecture comme base pour le développement de celui-ci.  

Afin de pouvoir utiliser CubicWeb pour la réalisation d'applications 	performantes et qui pouvoir utiliser l'informatique dans nuages pour les héberger, celles-ci doivent respecter les principes fondateurs du web, principalement l'architecture REST.


\section{Objectifs}
Les objectifs de stage sont :
\begin{itemize}
	\item dresser l'état des lieux du respect des principes de l'architecture REST. Il s'agit de rédiger un document expliquant les points de l'architecture REST qui ne sont pas respectés par les applications CubicWeb.
	
	\item implémenter dans CubicWeb et dans Pyramid-CubicWeb des solutions permettant le respect de toutes les contraintes de l'architecture REST.  Les solutions implémentées doivent être compatibles avec les anciennes versions et ne doivent pas introduire de grand changement dans l'utilisation de CubicWeb.
	
	\item le dernier objectif est de vérifier le fonctionnement des applications existantes avec les nouvelles modifications et s'assurer qu'aucune régression n'a été introduite. 
\end{itemize}